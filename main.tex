\documentclass{beamer}

% ===============================================================
% BEGIN SETUP
% ===============================================================

\usetheme{metropolis}
\usefonttheme{serif}
\usepackage[skip=0pt]{caption}

\captionsetup{labelformat=empty,labelsep=none}

\setbeamertemplate{footline}
{
  \leavevmode%
  \hbox{%
    \begin{beamercolorbox}[wd=.8\paperwidth,ht=2.25ex,dp=1ex,center]{author in head/foot}%
    \end{beamercolorbox}%
    \begin{beamercolorbox}[wd=.2\paperwidth,ht=2.25ex,dp=1ex,center]{title in head/foot}%
      \insertframenumber{} \hspace*{1ex}
    \end{beamercolorbox}
  }%
  \vskip0pt%
}
\setbeamertemplate{navigation symbols}{}

\setlength{\fboxsep}{0pt}
\setlength{\fboxrule}{1pt}

\newcommand{\image}[1]{\begin{figure}\fbox{\includegraphics[width=\linewidth,height=0.78\textheight,keepaspectratio]{./images/#1.pdf}}\end{figure}}

\newcommand{\htimage}[2]{\begin{figure}\fbox{\includegraphics[width=\linewidth,height=#2,keepaspectratio]{./images/#1.pdf}}\end{figure}
}

\newcommand{\capimage}[2]{\begin{figure}\caption{#2}\fbox{\includegraphics[width=\linewidth,height=0.70\textheight,keepaspectratio]{./images/#1.pdf}}\end{figure}}

% ===============================================================
% END SETUP
% ===============================================================

\begin{document}
\title{Auto Clustering TensorFlow Graphs}   
\author{Sanjoy Das}
\institute{Google}
\date{January 24, 2019}

\frame{\titlepage} 


\begin{frame}[fragile]
\frametitle{Quick TensorFlow Primer} 
  \begin{columns}
    \begin{column}{0.48\textwidth}
      \begin{itemize}
      \item Dataflow graph executor
      \item Explicit control (dashed) and data (solid) edges
      \item Supports an open set of operations
      \item Operations can have side effects
      \item Can represent loops and conditionals
      \end{itemize}
    \end{column}
    \begin{column}{0.48\textwidth}
      \image{tensorflow-primer}
    \end{column}
  \end{columns}
\end{frame}


\begin{frame}[fragile]
\frametitle{The TensorFlow/XLA Bridge In Action}
  \begin{columns}
    \begin{column}{0.23\textwidth}
      \image{tf-xla-bridge-in-action-a}
    \end{column}
    \begin{column}{0.23\textwidth}
      \image{tf-xla-bridge-in-action-b}
    \end{column}
    \begin{column}{0.45\textwidth}
      \image{tf-xla-bridge-in-action-c}
    \end{column}
  \end{columns}
\end{frame}


\begin{frame}[fragile]
\frametitle{The TensorFlow/XLA Bridge}
  \begin{itemize}
  \item Decides which parts should be compiled by XLA (Clustering)
  \item Converts TensorFlow nodes into XLA subgraphs (Translator)
  \item Compiles and executes a TensorFlow subgraph using XLA (JIT)
  \end{itemize}
\end{frame}


\begin{frame}[fragile]
\frametitle{The TensorFlow/XLA Bridge: Translator}
  \begin{itemize}
  \item Maps one TensorFlow node into one or more XLA nodes
  \item Not all TensorFlow ops are supported
  \item Interesting area for IR design
  \end{itemize}
\end{frame}


\begin{frame}[fragile]
\frametitle{The TensorFlow/XLA Bridge: Translator}
  \begin{columns}
    \begin{column}{0.48\textwidth}
      For example \texttt{Y = tf.SoftMax(X)} node is lowered into (roughly) the XLA graph shown on the right:
    \end{column}
    \begin{column}{0.48\textwidth}
      \image{softmax-lowering}
    \end{column}
  \end{columns}
\end{frame}


\begin{frame}[fragile]
  \frametitle{The TensorFlow/XLA Bridge: JIT}
  \begin{itemize}
  \item TensorFlow invokes XLA as a ``Just In Time'' Compiler
  \item Key functionality in the \texttt{\_XlaCompile} and \texttt{\_XlaRun} op kernels
  \item Does some runtime specialization because XLA needs compile-time constant shapes
  \item Implements ``lazy compilation''
  \end{itemize}
\end{frame}

\begin{frame}[fragile]
  \frametitle{The TensorFlow/XLA Bridge: ``Auto'' Clustering}
  \begin{itemize}
  \item Automatically discover clusters that should be compiled by XLA
  \item Should always preserve graph semantics
  \item Performance compared to TensorFlow should be never be worse and often be better
  \end{itemize}
\end{frame}


\begin{frame}[fragile]
  \frametitle{The TensorFlow/XLA Bridge: ``Auto'' Clustering}
  \large{\textbf{So ... what's the big deal?}}
\end{frame}


\begin{frame}[fragile]
  \frametitle{The TensorFlow/XLA Bridge: ``Auto'' Clustering}
  \large{\textbf{Auto clustering is surprisingly difficult!}}
\end{frame}


\begin{frame}[fragile]
  \frametitle{Auto Clustering: Cycle Detection}
  \begin{columns}
    \begin{column}{0.31\textwidth}
      \image{cycle-detection-a}
    \end{column}
    \begin{column}{0.31\textwidth}
      \image{cycle-detection-b}
    \end{column}
    \begin{column}{0.31\textwidth}
      \image{cycle-detection-c}
    \end{column}
  \end{columns}
\end{frame}

\begin{frame}[fragile]
  \frametitle{Auto Clustering: Cycle Detection}
  \begin{itemize}
  \item Online cycle detection algorithm
  \item Run as we make decisions about which nodes to put in which cluster
  \item Uses a worklist because the technique is visit order dependent
    \htimage{cycle-detection-d}{0.4\textheight}
  \end{itemize}
\end{frame}

\begin{frame}[fragile]
  \frametitle{Conditionals in TensorFlow}

  \begin{columns}
    \begin{column}{0.48\textwidth}
      \image{switch-a}
    \end{column}
    \begin{column}{0.48\textwidth}
      \image{switch-b}
    \end{column}
  \end{columns}
\end{frame}

\begin{frame}[fragile]
  \frametitle{Conditionals in TensorFlow}

  \begin{columns}
    \begin{column}{0.48\textwidth}
      \image{merge-a}
    \end{column}
    \begin{column}{0.48\textwidth}
      \image{merge-b}
    \end{column}
  \end{columns}
\end{frame}

\begin{frame}[fragile]
  \frametitle{Conditionals in TensorFlow}

  \image{control-flow-in-tf-a}
\end{frame}

\begin{frame}[fragile]
  \frametitle{Conditionals in TensorFlow}

  \image{control-flow-in-tf-b}
\end{frame}

\begin{frame}[fragile]
  \frametitle{Auto Clustering \& Deadness}
  \begin{columns}
    \begin{column}{0.48\textwidth}
      \image{zombie-nodes-a}
    \end{column}
    \begin{column}{0.48\textwidth}
      \image{zombie-nodes-b}
    \end{column}
  \end{columns}
\end{frame}

\begin{frame}[fragile]
  \frametitle{Auto Clustering \& Deadness}
  \begin{itemize}
  \item Map each node to a symbolic predicate that is true iff the node is executed
  \item All nodes in the same cluster are constrained to have the same ``is live'' predicate
  \item Conservatively correct because we check syntactic equivalence
  \end{itemize}
\end{frame}

\begin{frame}[fragile]
  \frametitle{Serializing Blocking Operations [Work In Progress]}
  \begin{columns}
    \begin{column}{0.32\textwidth}
      \image{serialize-blocking-operations-a}
    \end{column}
    \begin{column}{0.42\textwidth}
      \image{serialize-blocking-operations-b}
    \end{column}
    \begin{column}{0.22\textwidth}
      \image{serialize-blocking-operations-c}
    \end{column}
  \end{columns}
\end{frame}

\begin{frame}[fragile]
  \frametitle{Auto Clustering: Device To Host Copies}

  \begin{columns}
    \begin{column}{0.48\textwidth}
      \begin{itemize}
      \item XLA only produces device memory buffers
      \item May introduce bottlenecks by not letting the CPU run ahead of the GPU
      \item We ``decluster'' nodes to avoid this problem
      \end{itemize}
    \end{column}
    \begin{column}{0.48\textwidth}
      \image{device-to-host-copies}
    \end{column}
  \end{columns}
\end{frame}

%% \begin{frame}[fragile]
%%   \frametitle{Runtime Specialization of Shapes}
%%   \begin{itemize}
%%   \item XLA only supports static shapes today.
%%   \item The XLA executable must be versioned on the input shapes.
%%   \item In some cases we also need to version on input \textit{values}.
%%   \end{itemize}
%% \end{frame}

\begin{frame}[fragile]
  \frametitle{Runtime Specialization of Shapes}

  %% \begin{columns}
  %%   \begin{column}{0.48\textwidth}
  %%     \image{shape-specialization-b}
  %%   \end{column}
  %%   \begin{column}{0.48\textwidth}
  %%     \image{shape-specialization-c}
  %%   \end{column}
  %% \end{columns}

  \begin{itemize}
    \item XLA is statically shaped while TensorFlow is not
    \item We version (with caching) XLA executables on the shapes in the cluster
  \end{itemize}
\end{frame}

\begin{frame}[fragile]
  \frametitle{Runtime Specialization of Shapes}
  Output shape for \texttt{tf.slice(input, begin, size)}:
\begin{verbatim}
  output_size[i] =
    (size[i] == -1) ? (input.shape()[i] - begin[i]) :
                      size[i];
\end{verbatim}
\end{frame}


\begin{frame}[c]
  \frametitle{Reducing Unnecessary Recompilation}

  \image{reduce-compilation-symbolic-slice}
\end{frame}

\begin{frame}[fragile]
  \frametitle{Reducing Unnecessary Recompilation}

  \image{reduce-compilation-a}
\end{frame}

\begin{frame}[fragile]
  \frametitle{Reducing Unnecessary Recompilation}

  \image{reduce-compilation-e}
\end{frame}

\begin{frame}[fragile]
  \frametitle{Reducing Unnecessary Recompilation}

  \image{reduce-compilation-f}
\end{frame}

\begin{frame}[fragile]
  \frametitle{Reducing Unnecessary Recompilation}

  \image{reduce-compilation-symbolic-slice-fixed}
\end{frame}

\begin{frame}[fragile]
  \frametitle{Resource Variable Operations in TensorFlow}
  \begin{columns}
    \begin{column}{0.58\textwidth}
      \begin{itemize}
      \item Resource variables are mutable ``cells'' that point to immutable tensors
      \item Semantically, reads and writes execute in a total order consistent with the partial order of the graph
      \item Given the graph on the right we can assert ``\texttt{r0 == 2} implies \texttt{r1 == 1}''
      \end{itemize}
    \end{column}
    \begin{column}{0.38\textwidth}
      \image{resource-var-a}
    \end{column}
  \end{columns}
\end{frame}


\begin{frame}[fragile]
  \frametitle{Resource Variable Operations in XLA}

  \begin{itemize}
  \item Clustering resource variable operations can be important in some cases.

  \item However, XLA would prefer not to represent resource variable operations directly in its IR.

  \item Solution:
    \begin{itemize}
    \item Split the computation into ``pure'' and ``impure'' (side effecting) parts
    \item Have XLA handle the ``pure'' bits, and the TF/XLA bridge handle the ``impure'' bits
    \end{itemize}
  \end{itemize}
\end{frame}

\begin{frame}[fragile]
  \frametitle{Resource Variable Operations in XLA}
  \begin{columns}
    \begin{column}{0.28\textwidth}
\begin{verbatim}
r0 = Read(X)
r1 = Read(Y)
Write(42, Z)
r2 = Read(Z)
r3 = r0 + r1 + r2
Write(Z, r3)
\end{verbatim}
    \end{column}
    \vrule{}
    \begin{column}{0.70\textwidth}
      \begin{enumerate}
        \item
\begin{verbatim}
// The TF/XLA Bridge
rX0 = Read(X); rY0 = Read(Y)
\end{verbatim}
        \item
\begin{verbatim}
// The XLA Computation
r0 = rX0  // Read(X)
r1 = rY0  // Read(Y)
rZ0 = 42  // Write(42, Z)
r2 = rZ0  // Read(Z)
r3 = r0 + r1 + r2
rZ1 = r3  // Write(Z, r3)
\end{verbatim}
        \item
\begin{verbatim}
// The TF/XLA Bridge
Write(Z, rZ1)
\end{verbatim}
      \end{enumerate}
    \end{column}
  \end{columns}
\end{frame}


\begin{frame}[fragile]
  \frametitle{Resource Variable Operations in XLA}
  \begin{columns}
    \begin{column}{0.48\textwidth}
      \htimage{resource-var-b}{0.6\textheight}
      Assert ``\texttt{*X == 6} implies \texttt{r0 == 8}''
    \end{column}
    \begin{column}{0.48\textwidth}
      \image{resource-var-c}
    \end{column}
  \end{columns}
\end{frame}


\begin{frame}[fragile]
  \frametitle{Resource Variable Operations in XLA}
  \begin{itemize}
  \item Solution: Static Analysis!
  \item Analyze the TensorFlow graph to figure out which pairs of resource operations cannot be put into the same cluster
  \item Make auto-clustering respect these constraints
  \end{itemize}
\end{frame}


\begin{frame}[fragile]
  \frametitle{Auto Clustering: Current Status}
  \begin{itemize}
  \item We've made significant progress towards auto-clustering for XLA GPU, but we're not production ready yet
  \item We'd love for you to try it out!
    \begin{itemize}
    \item Change the \texttt{TF\_XLA\_FLAGS} environment variable to include \texttt{--tf\_xla\_auto\_jit=2} to enable auto clustering for all graphs
    \item You may have to change your model to use resource variables for best results
    \end{itemize}
  \item There are no immediate plans for auto-clustering for XLA CPU
  \end{itemize}
\end{frame}

\begin{frame}[fragile]
  \frametitle{Thank You}

  \Huge{Thank you!}

  \Huge{Questions?}
\end{frame}

\end{document}
